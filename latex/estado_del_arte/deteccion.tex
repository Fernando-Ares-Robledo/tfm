La detección de ransomware se puede realizar mediante diversos métodos, cada uno con sus particularidades y eficacia frente a diferentes tipos de ataques. A continuación, se describen los principales métodos utilizados en la actualidad para detectar y mitigar el impacto del ransomware.

\begin{enumerate}
  \item \textbf{Detección basada en firmas.} Identifica ransomware mediante la búsqueda de patrones específicos en su código, comparándolos con una base de datos de firmas conocidas. Si se encuentra una coincidencia, se activan medidas de mitigación\autocite{pastorino}.
  
  \item \textbf{Detección basada en el comportamiento.} Observa las acciones del software, como la rápida encriptación de archivos o la comunicación con servidores de control. Detecta actividades sospechosas y toma medidas preventivas, siendo más efectiva contra nuevas variantes de ransomware y ataques de día cero\autocite{fhabte2023ransomware}.
  
  \item \textbf{Detección basada en engaños.} Consiste en crear sistemas falsos o datos que aparentan ser vulnerables al ransomware. Cuando el ransomware ataca estos señuelos, se genera una alerta, permitiendo tomar medidas para detener el ataque y analizar el comportamiento del malware. Esta estrategia es valiosa para comprender las tácticas de los atacantes y fortalecer las defensas de seguridad\autocite{pedro2022cripto}.
  
  \item \textbf{Detección por tráfico anormal.} Se centra en analizar el flujo de datos en una red para identificar actividades inusuales que podrían señalar un ataque de ransomware. Esto implica estar atento a cambios repentinos en el volumen de datos, comunicaciones hacia destinos no habituales o el uso de protocolos poco comunes asociados con ransomware. Al detectar estas anomalías, se pueden implementar medidas de protección para detener el ataque y salvaguardar los sistemas comprometidos\autocite{fhabte2023ransomware}.
\end{enumerate}

