
Existe una gran variedad de soluciones comerciales diseñadas para combatir el ransomware. Estas soluciones varían en enfoque, capacidades y nivel de protección ofrecido, adaptándose a diferentes necesidades y presupuestos. A continuación, se describen algunos de los tipos más comunes de soluciones anti-ransomware disponibles en el mercado.

\begin{enumerate}
  \item \textbf{Antivirus/Antimalware Avanzado:} Estas soluciones deben ser regularmente actualizadas para incorporar correcciones de seguridad, mejoras en la detección y protección contra nuevas amenazas. Existen opciones diseñadas para usuarios particulares, ofreciendo soluciones más básicas, económicas e incluso gratuitas, como AVG. También hay disponibles en el mercado opciones de pago, que incluyen soporte técnico y funciones más avanzadas, tales como Panda Premium, Norton 360, McFee, Kaspersky Anti-Ransomware o Malwarebytes Premium. Actualmente, algunos antivirus incorporan asistencia de Inteligencia Artificial, conocidos como Next-Gen Antivirus (NGAV).

  \item \textbf{Firewalls y Gateways de Seguridad:} Las soluciones de firewall y los gateways de seguridad están diseñados para filtrar el tráfico malicioso y bloquear intentos de ataques de ransomware antes de que lleguen a la red interna de una organización. Un ejemplo de esto son los firewalls de alto rendimiento de la serie Quantum Force Gateway.

  \item \textbf{Backup y Recuperación de Datos:} Mantener un sistema de backup robusto es fundamental para la recuperación de datos en caso de un ataque de ransomware. Estas soluciones permiten a las organizaciones restaurar sus sistemas a un estado anterior sin pagar el rescate.

  \item \textbf{Seguridad de Endpoint:} Las soluciones de seguridad de endpoint están diseñadas para proteger los dispositivos finales, como computadoras portátiles, estaciones de trabajo y dispositivos móviles. Pueden detectar y bloquear actividades maliciosas, incluido el ransomware, en estos dispositivos. Ejemplos de estas herramientas incluyen CyberArk Endpoint Privilege Manager (EPM) y Endpoint Detection and Response (EDR).
\end{enumerate}
