 La ENISA\autocite{enisa2022ransomware} proporciona una serie de recomendaciones sobre cómo actuar ante los ataques de ransomware, enfocándose en la resiliencia contra estos ataques y la respuesta una vez que ocurren como son:

\begin{itemize}
  \item Mantener copias de seguridad, siguiendo la regla de respaldo 3-2-1.
  \item Cifrar los datos personales de acuerdo con el GDPR y controlar los riesgos adecuadamente.
  \item Utilizar software de seguridad que pueda detectar la mayoría de los ransomware.
  \item Mantener actualizadas las políticas de seguridad y privacidad, practicando una buena higiene de seguridad (segmentación de red, parches al día, respaldos regulares, gestión de identidad y acceso con MFA).
  \item Realizar evaluaciones de riesgo regularmente y considerar la contratación de un seguro contra ransomware.
  \item Restringir los privilegios administrativos, aplicando el Principio de Menor Privilegio.
  \item Familiarizarse con las agencias gubernamentales locales que brindan asistencia en incidentes de ransomware y definir protocolos de actuación en caso de ataque.
\end{itemize}

Además, en caso de ataque exitoso recomienda:

\begin{itemize}
  \item No pagar el rescate ni negociar con los actores de la amenaza.
  \item Poner en cuarentena los sistemas afectados para contener la infección y evitar que se propague.
  \item Consultar iniciativas como The No More Ransom Project de Europol, que puede descifrar varias variantes de ransomware.
  \item Compartir información sobre el incidente de ransomware con las autoridades.
\end{itemize}

INCIBE\autocite{incibe_nd_ayuda} por su parte también ofrece recomendaciones para la resiliencia contra el ransomware, enfatizando la importancia de mantener copias de seguridad siguiendo la regla 3-2-1, el cifrado de datos personales, y la implementación de software de seguridad. En caso de ataque, aconseja no pagar el rescate, poner en cuarentena los sistemas afectados, y buscar ayuda a través de iniciativas como The No More Ransom Project. Ademas también proporciona unas guías\autocite{incibe_nd_actuar} para intentar eliminar el ransomware que se puede resumir en lo siguiente:
\begin{itemize}
    \item Arranca el ordenador en modo seguro con opciones de red.
    \item Utiliza una herramienta de limpieza para deshacerte del ransomware.
    \item Ejecuta un análisis adicional para asegurar que el sistema está completamente limpio.
    \item Recupera los archivos que fueron encriptados por el ransomware.
\end{itemize}