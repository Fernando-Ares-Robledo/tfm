\subsubsection{Marcos Legales de Protección de Datos}

Los marcos legales diseñados para regular y proteger la privacidad y seguridad de los datos personales de los individuos incluyen el Reglamento General de Protección de Datos (GDPR) en la Unión Europea y la Ley Orgánica 3/2018 de protección de datos personales y garantía de los derechos digitales (LOPDGDD) del 5 de diciembre en España. Estas leyes establecen principios y requisitos para el procesamiento de datos personales por parte de organizaciones y empresas, incluidas medidas de seguridad para prevenir la pérdida, el acceso no autorizado o la divulgación de datos.

\subsubsection{Consecuencias de un Ataque de Ransomware}

Un ataque de ransomware contra cualquier entidad, ya sea una organización o empresa, implica consecuencias no solo económicas, sino también responsabilidades legales derivadas del deber de protección de datos. Las organizaciones afectadas pueden enfrentar sanciones legales, demandas civiles y daños a su reputación debido a la falta de seguridad adecuada y la protección insuficiente de los datos.

\subsubsection{El Derecho a la Protección de Datos}

El bien jurídico protegido en todos los delitos informáticos es el derecho a la protección de datos de carácter personal, reconocido en el artículo 18.4 de la Constitución Española y por el Tribunal Constitucional en STC 292/2000, así como en el preámbulo de la LOPDGDD.

\subsubsection{Responsable del Tratamiento de Datos}

Tanto el Reglamento General de Protección de Datos (GDPR) en la Unión Europea como la Ley Orgánica de Protección de Datos (LOPDGD) en el Título V en España  se regula la figura del Responsable del Tratamiento de datos en una empresa u organización. Este responsable es la persona o entidad que determina los fines y medios del tratamiento de datos personales. Sus responsabilidades incluyen garantizar el cumplimiento de las normativas de protección de datos, implementar medidas de seguridad adecuadas para proteger la información, y responder a las solicitudes de los titulares de datos sobre sus derechos ARCO (acceso, rectificación, cancelación y oposición). Es además el responsable de notificar las brechas de seguridad de datos a la autoridad de protección de datos y, en ciertos casos, a los individuos afectados..

\subsubsection{Procedimientos y Sanciones}

En el título VIII de la LOPDGD se regula los procedimientos a seguir en caso de incumplimiento de la normativa de la protección de datos, y en el Título IX la potestad de imposición de sanciones administrativas según el tipo de infracción realizada. Las empresas también pueden recibir reclamaciones civiles por parte de los clientes por la pérdida de datos, regulado en los Artículos 1101 y 1902 del Código Civil.

\subsubsection{Denuncia de Ataques de Ransomware}

Cualquier ataque de ransomware debe ser denunciado a las fuerzas de seguridad del estado, ya que comprende delitos penalmente castigados como daños informáticos, blanqueo de capitales, intrusismo, estafa, pertenencia a organización criminal y delitos contra la intimidad.
