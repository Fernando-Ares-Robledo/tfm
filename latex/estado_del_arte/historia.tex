\subsubsection{El génesis (1986-2005)}
El ransomware ha emergido como una de las amenazas cibernéticas más devastadoras,
comenzando su historia a finales de los 80 con el ransomware AIDS, creado por el Dr. Joseph Popp,  que cifraba las máquinas tras un número determinado de reinicios, exigiendo un rescate de 189 dólares\autocite{bates1990aids}.

\subsubsection{Aumento de la propagación y sofisticación (2005-2010)}
Con la aparicion de internet y su comienzo en popularidad  el ransomware comenzó a adoptar métodos más sofisticados tanto en términos de ataque como de exigencia de pagos. Se observa un cambio en las técnicas de distribución, pasando de los disquetes y correos electrónicos a aprovechar vulnerabilidades de seguridad en el software y el uso de kits de explotación. El ransomware Gpcode [2], que reapareció en varias versiones cada vez más sofisticadas, fue uno de los primeros en utilizar técnicas avanzadas de cifrado.

El ransomware comienza a internacionalizarse, con ataques que no se limitan a regiones específicas sino que apuntan a usuarios de Internet en todo el mundo. El uso de tácticas de ingeniería social, como falsas advertencias de organismos de aplicación de la ley acusando a los usuarios de actividades ilegales y exigiendo el pago de "multas", se vuelve común.

En 2008 la variante de Gpcode conocida como AK utilizó cifrado RSA-1024, un salto significativo en la sofisticación del cifrado. Aunque los expertos en seguridad fueron capaces de contrarrestar algunas versiones anteriores, la complejidad del cifrado RSA-1024 presentó un desafío mucho mayor\autocite{knowbe4gpcode}. 

En los siguientes dos años aparecen más variantes de ransomware, incluidos WinLock y Reveton, que perfeccionan aún más el modelo de "policía falsa". Estos ransomwares no cifraban archivos pero restringían el acceso al sistema y mostraban mensajes alarmantes diseñados para intimidar a los usuarios para que pagaran.




\subsubsection{El Auge del Crypto-ransomware (2011-2016)}
En 2013 CryptoLocker se convierte en uno de los ransomware más infames, cifrando archivos de usuario con cifrado fuerte y exigiendo Bitcoin como pago. Su éxito marca el inicio de una ola de crypto-ransomware\autocite{naraine2013cryptolocker}. 

A partir del 2015 aparecen variantes aún más sofisticadas como Locky, Petya y TeslaCrypt, aprovechando la creciente popularidad de las criptomonedas para los pagos de rescate.

\subsubsection{Expansion global (2017-presente)}
WannaCry y NotPetya causan estragos a nivel mundial, afectando a cientos de miles de sistemas en más de 150 países. WannaCry explotaba una vulnerabilidad en Windows, mientras que NotPetya se disfrazaba como ransomware pero en realidad buscaba causar daño.

Comienza a observarse un aumento en los ataques dirigidos, en los que los atacantes se enfocan en organizaciones específicas, gobiernos, y sistemas críticos, exigiendo rescates mucho más elevados, como fue el ciberataque al Hospital Clínic de
Barcelona en 2023\autocite{blanchar2023ciberataque}

Los ataques de ransomware siguen evolucionando con tácticas como el "doble extorsión", donde los atacantes no sólo cifran archivos, sino que también roban datos y amenazan con publicarlos si no se paga el rescate.

