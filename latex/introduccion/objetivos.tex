\subsubsection{Objetivos Generales}

\begin{itemize}
  \item Desarrollar una comprensión básica de los mecanismos y efectos del ransomware, así como de la importancia de los backups en la recuperación de ataques de ransomware.
  \item Implementar una solución de backup con Bacula que demuestra ser efectiva en la recuperación de datos tras un ataque de ransomware.
\end{itemize}

\subsubsection{Objetivos Específicos}
\begin{itemize}
  \item Analizar las capacidades y configuraciones óptimas de Bacula para la protección contra ransomware.
  \item Implementar un entorno de prueba con Bacula.
  \item Desarrollar una guía de mejores prácticas para el uso de Bacula en la protección contra ransomware.
  \item Evaluar la viabilidad y eficiencia de la solución propuesta.
\end{itemize}

\subsubsection{Objetivo de Sostenibilidad y Ética}
\begin{itemize}
  \item 
Promover la importancia de la ciberseguridad y la protección de datos desde una perspectiva ética y de sostenibilidad reflexionando sobre cómo una gestión de backups efectiva contribuye a la seguridad de la información y la resiliencia organizacional, enfatizando la responsabilidad de proteger los datos de los usuarios y clientes.

\end{itemize}

\subsubsection{Objetivos de Entrega del Proyecto}
\begin{itemize}
  \item Entregar en tiempo y forma las entregas parciales.
  \item Desarrollar la memoria final del trabajo y la presentación en video.
\end{itemize}

\subsubsection{Objetivos que Debe Cumplir el Sistema Implementado}
\begin{itemize}
  \item Efectividad en la Recuperación; capacidad para recuperar datos específicos afectados por un escenario de ransomware simulado de manera efectiva.
  \item La solución implementada debe ser administrable por personal con conocimientos de TI.
  \item Documentación Detallada, incluyendo documentación clara sobre la configuración, operación y recuperación.
\end{itemize}
