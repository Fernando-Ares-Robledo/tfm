\subsubsection{Punto de partida}
En el mundo digital actual, los ataques de ransomware se han convertido en una de las amenazas más significativas y disruptivas para las organizaciones de todos los tamaños y sectores. Este tipo de malware cifra los archivos del sistema infectado, exigiendo un rescate a cambio de la clave de descifrado. La necesidad de proteger los datos críticos ante esta amenaza es más urgente que nunca, dado el aumento en la frecuencia y sofisticación de estos ataques.

Las consecuencias de un ataque de ransomware van más allá del impacto financiero directo relacionado con el rescate. Incluyen interrupciones operativas, pérdida de datos críticos, daños a la reputación y costos asociados con la recuperación del sistema. Aunque existen diversas estrategias de ciberseguridad para prevenir y mitigar estos ataques, la implementación efectiva de backups se ha establecido como una de las medidas más confiables y efectivas para recuperarse de un ataque de ransomware sin ceder ante las demandas de los atacantes.


\subsubsection{Aportación realizada}
El objetivo de este trabajo de fin de máster (TFM) es desarrollar una solución basada en Bacula, una herramienta de backup, recuperación y verificación de datos, para crear una estrategia de protección de datos robusta y eficiente contra el ransomware. A través de la implementación práctica y la evaluación de esta solución, se busca:

\begin{itemize}
    \item Demostrar la eficacia de los backups como medida de recuperación ante ataques de ransomware, minimizando la pérdida de datos y el impacto operacional.
    \item Explorar cómo la configuración y gestión de Bacula pueden optimizarse para enfrentar específicamente el desafío del ransomware, identificando mejores prácticas en la programación de backups, la retención de datos y la recuperación rápida y eficaz.
    \item Contribuir al campo de la ciberseguridad con un estudio aplicado y soluciones prácticas que puedan ser adoptadas por organizaciones para fortalecer su resiliencia ante ataques de ransomware.
\end{itemize}

Este trabajo no solo se propone demostrar la viabilidad técnica de la solución propuesta, sino también enfatizar la importancia de una estrategia de backups bien planificada y ejecutada como parte integral de la defensa contra el ransomware. A través de este enfoque, el TFM contribuirá al desarrollo de conocimientos y herramientas prácticas que puedan ser utilizadas para proteger los activos digitales esenciales en un entorno cada vez más amenazado por el ransomware.