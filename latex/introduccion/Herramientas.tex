
Para el desarrollo y documentación de este proyecto, se han empleado diversas herramientas tecnológicas, seleccionadas cuidadosamente para optimizar la eficiencia, colaboración y gestión de los recursos disponibles. Estas herramientas se han agrupado en función de su funcionalidad principal:

\subsubsection{Gestión de Proyectos y Colaboración}
\begin{itemize}
    \item \textbf{Trello:} Una aplicación basada en la web para la gestión de proyectos que utiliza el método Kanban. Permite a los equipos organizar tareas, establecer plazos y colaborar en diferentes fases del proyecto.
    \item \textbf{Google Documents:} Una suite de oficina en línea que facilita la creación, edición y almacenamiento compartido de documentos de texto, hojas de cálculo y presentaciones, permitiendo la colaboración en tiempo real entre los usuarios.
\end{itemize}

\subsubsection{Desarrollo y Control de Versiones}
\begin{itemize}
    \item \textbf{GitHub:} Una plataforma de hospedaje de código que utiliza Git para el control de versiones, facilitando la colaboración en proyectos de software mediante la gestión de ramas, seguimiento de problemas y revisión de código.
\end{itemize}

\subsubsection{Virtualización y Sistemas Operativos}
\begin{itemize}
    \item \textbf{VirtualBox:} Un software de virtualización de código abierto que permite ejecutar múltiples sistemas operativos simultáneamente en una sola máquina física.
    \item \textbf{Debian Bookworm:} La versión estable de Debian utilizada para servidores, conocida por su estabilidad y seguridad.
    \item \textbf{Windows Server:} Un sistema operativo diseñado por Microsoft enfocado en la gestión de servidores, ofreciendo herramientas para soportar infraestructuras empresariales y aplicaciones web.
\end{itemize}

\subsubsection{Creación de Contenidos y Presentaciones}
\begin{itemize}
    \item \textbf{OBS (Open Broadcaster Software):} Un software libre y de código abierto para grabación de vídeo y transmisión en vivo, ampliamente utilizado para crear contenido multimedia.
    \item \textbf{Overleaf:} Una herramienta en línea para la escritura de documentos LaTeX en tiempo real, permitiendo la colaboración entre varios autores y la compilación de documentos sin necesidad de instalar software adicional.
    \item \textbf{PowerPoint:} Un programa de presentación desarrollado por Microsoft que se utiliza para crear diapositivas dinámicas y visuales, facilitando la comunicación de ideas y resultados de proyectos.
\end{itemize}

\subsubsection{Backup y Recuperación de Datos}
\begin{itemize}
    \item \textbf{Bacula:} Un conjunto de programas de software libre y de código abierto que permiten administrar la copia de seguridad, recuperación y verificación de datos a través de una red de computadoras.
\end{itemize}
