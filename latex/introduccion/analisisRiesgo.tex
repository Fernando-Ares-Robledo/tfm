


\definecolor{riskcolor}{RGB}{225,225,250} % Fondo claro para los riesgos
\definecolor{highprob}{RGB}{255,200,200} % Color para probabilidad alta
\definecolor{medprob}{RGB}{255,255,200} % Color para probabilidad media
\definecolor{lowprob}{RGB}{200,255,200} % Color para probabilidad baja

\newcommand{\alta}{\textcolor{red}{\ding{55}}}
\newcommand{\media}{\textcolor{orange}{\ding{51}}}
\newcommand{\baja}{\textcolor{green}{\ding{108}}}

\tcbset{
    riskbox/.style={
        colback=riskcolor,
        colframe=black,
        fonttitle=\bfseries,
        colbacktitle=riskcolor,
        coltitle=black,
        enhanced,
        breakable,
        left=4mm,
        right=4mm,
        top=2mm,
        bottom=2mm,
        boxsep=2mm
    },
    prob/.style={circle, minimum width=1cm, minimum height=1cm, draw=black, fill=#1, align=center},
    probhigh/.style={prob=highprob},
    probmed/.style={prob=medprob},
    problow/.style={prob=lowprob}
}

\begin{tcolorbox}[riskbox, title=Riesgo 1: Retrasos en el Cronograma]
\textbf{Probabilidad:} Alta \alta; \\
\textbf{Impacto:} Medio \media\\
\textbf{Mitigación:} Adoptar una metodología ágil permite ajustes flexibles y reasignación de recursos para mantener el proyecto en curso. Las revisiones regulares y la planificación de sprints ayudan a identificar retrasos tempranamente.
\end{tcolorbox}

\begin{tcolorbox}[riskbox, title=Riesgo 2: Limitaciones Técnicas con Bacula]
\textbf{Probabilidad:} Medio \media; \\
\textbf{Impacto:} Alto \alta\\
\textbf{Mitigación:} Invertir tiempo en la formación sobre Bacula y participar en comunidades o foros relacionados para buscar soporte. Realizar pruebas preliminares para identificar limitaciones técnicas antes de la fase de implementación.
\end{tcolorbox}

\begin{tcolorbox}[riskbox, title=Riesgo 3: Dificultades en la Simulación de Ransomware]
\textbf{Probabilidad:} Medio \media; \\
\textbf{Impacto:} Alto \alta\\
\textbf{Mitigación:} Utilizar herramientas de simulación de ataques reconocidas y seguras que permitan simular el comportamiento del ransomware sin infringir leyes o comprometer sistemas. Consultar con el tutor y expertos en ciberseguridad para asegurar que las pruebas sean éticas y legales.
\end{tcolorbox}

\begin{tcolorbox}[riskbox, title= Riesgo 4: Problemas de Integración de Herramientas]
\textbf{Probabilidad:} Bajo \baja; \\
\textbf{Impacto:} Medio \media\\
\textbf{Mitigación:} Seleccionar herramientas y tecnologías compatibles desde el inicio. Realizar pruebas de integración tempranas para detectar y solucionar problemas de compatibilidad.
\end{tcolorbox}

\begin{tcolorbox}[riskbox, title= Riesgo 5: Pérdida de Datos o Fallos en el Entorno de Pruebas]
\textbf{Probabilidad:} Bajo \baja; \\
\textbf{Impacto:} Alto \alta\\
\textbf{Mitigación:}  Implementar prácticas de backup regulares del entorno de pruebas y utilizar control de versiones para el código y la documentación. Esto incluye el uso de GitHub para almacenamiento y seguimiento
\end{tcolorbox}

\begin{tcolorbox}[riskbox, title= Riesgo 6: Cambios en los Requisitos o Alcance del Proyecto]
\textbf{Probabilidad:} Medio \media\\
\textbf{Impacto:} Medio \media\\
\textbf{Mitigación:}   Mantener una comunicación clara y continua con el tutor y revisar regularmente los objetivos del proyecto para adaptar el alcance si es necesario. La flexibilidad inherente de la metodología ágil facilita la gestión de cambios.
\end{tcolorbox}

\begin{tcolorbox}[riskbox, title= Riesgo 7. Acceso Limitado a Expertos o Recursos]
\textbf{Probabilidad:} Bajo \baja\\
\textbf{Impacto:} Medio \media\\
\textbf{Mitigación:}   Establecer contactos con expertos y comunidades en línea desde las etapas iniciales del proyecto. Planificar adecuadamente el uso de recursos disponibles y buscar alternativas o soporte adicional si es necesario.
\end{tcolorbox}