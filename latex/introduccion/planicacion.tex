La planificación se divide en sprints de 2 semanas, ajustándose a las fechas de las PECs y otros hitos clave del proyecto. A continuación, se presenta la planificación temporal: 

\begin{tcolorbox}[phasebox, title=Fases del trabajo]
    % Sprint 1
    \small
    \begin{tcolorbox}[sprintbox, title=Sprint 1: Preparación y presentación del Plan de Trabajo]
        \textbf{Tareas:}
        \begin{itemize}
            \item Definir el alcance del TFM y los objetivos específicos.
            \item Selección inicial de herramientas y recursos necesarios.
            \item Planificación inicial usando una herramienta de gestión de proyectos (Trello).
            \item Creación de un repositorio en GitHub.
        \end{itemize}
        \textbf{Recursos y Herramientas:}
        \begin{itemize}
            \item Documentación oficial de Bacula.
            \item Trello, Google Documents, GitHub.
            \item Google Scholar y bases de datos académicas.
        \end{itemize}
        \textbf{Retrospectiva:} Revisión de la planificación inicial, ajustes según feedback del tutor. 11 Mar. - 12 Mar.\\
        \textbf{Hito:} PEC 1- Plan de Trabajo entregado.
    \end{tcolorbox}


    \begin{tcolorbox}[sprintbox, title= Sprint 2 y 3: Análisis de requisitos y diseño inicial.]
    
        \textbf{Tareas:}
        \begin{itemize}
            \item Realizar una revisión exhaustiva del estado del arte en ransomware y estrategias de backup 13 Mar. - 20 Mar.
            \item Análisis de requisitos específicos para la implementación con Bacula 21 Mar. - 27 Mar.
            \item Diseño de la arquitectura de prueba e instalación de los requisitos necesarios. 28 Mar. - 3 Abr.
            
        \end{itemize}
        \textbf{Recursos y Herramientas:}
        \begin{itemize}
            \item Imágenes de instalación de Debian Bookworm y Windows server.

            \item Software de virtualización VirtualBox para crear entornos de pruebas seguros.

        \end{itemize}
        \textbf{Retrospectiva:} Evaluación de la comprensión del problema y del diseño propuesto  7 Abr. - 9 Abr.\\
        \textbf{Hito:} PEC 2 entregado con análisis y diseño preliminar 9 Abr.

    \end{tcolorbox}


    \begin{tcolorbox}[sprintbox, title= Sprint 4 a 6: Configuración de Bacula y pruebas]
        \textbf{Tareas:}
        \begin{itemize}
            \item Configuración de Bacula en el entorno de prueba  10 Abr. - 17 Abr.

            \item Ejecución de pruebas de restauración de backups en escenarios de desaste simulados 18 Abr. - 24 Abr.

            \item Reajustes en la configuración de Bacula basados en los resultados de las pruebas 25 Abr. - 1 May.

            
        \end{itemize}
        \textbf{Recursos y Herramientas:}
        \begin{itemize}
            \item Software Bacula y documentación relacionada.
            \item GitHub para documentar el proceso y resultados.

        \end{itemize}
        \textbf{Retrospectiva:} Revisión de la eficacia de las estrategias de backup y restauración implementadas 5 May. - 7 May.\\
        \textbf{Hito:} PEC 3 entregado con resultados de implementación y pruebas 7 May.

    \end{tcolorbox}

    
    \begin{tcolorbox}[sprintbox, title= Sprint 7 y 8: Documentación final y preparación de la memoria.]
        \textbf{Tareas:}
        \begin{itemize}
            \item Redacción detallada de la memoria, incluyendo metodología, resultados, y análisis de las pruebas 8 May. - 15 May.
            \item Preparación de la presentación final y el vídeo para la defensa 16 May. - 21 May.
        \end{itemize}
        
        \textbf{Recursos y Herramientas:}
        \begin{itemize}
            \item Google Documents/LaTeX para la elaboración de la memoria.
            \item Software para grabar video OBS 
            \item PowerPoint/Google Presentaciones para la elaboración de la presentación
            \item Directrices y normativa del TFM proporcionadas por la universidad para asegurar el cumplimiento en la presentación y documentación.

        \end{itemize}
        \textbf{Retrospectiva:} Evaluación completa del proyecto, ajustes finales de la documentación 21 May. - 10 Jun.\\
        \textbf{Hito:}PEC 4 entregado con la memoria final 10 Jun.

    \end{tcolorbox}
    
    \begin{tcolorbox}[sprintbox, title= Sprint 9: Preparación para la defensa.]
        \textbf{Tareas:}
        \begin{itemize}
            \item Finalización de la presentación del TFM 12 Jun. - 15 Jun.
            \item Preparación y grabación del vídeo de presentación 16 Jun. - 19 Jun.

            \item Ensayos de la defensa del TFM 15 Jun. - 24 Jun.

            
        \end{itemize}
       
        \textbf{Retrospectiva:}  Revisión final y ensayo de la presentación y defensa.\\
        \textbf{Hito:} PV entregado y defensa del TFM.

    \end{tcolorbox}

\end{tcolorbox}
\medskip

El diagrama de Grantt es el siguiente:

\begin{landscape}
\begin{ganttchart}[
    hgrid,
    vgrid={*1{blue, dashed}},
    x unit=0.5cm,
    y unit title=0.7cm,
    y unit chart=0.7cm,
    title/.append style={draw=none, fill=blue!20},
    title label font=\bfseries\footnotesize,
    bar/.append style={fill=blue!30},
    bar height=0.6,
    group right shift=0,
    group top shift=0.7,
    group height=.3,
    group peaks height=.2
    ]{1}{24}
    % Labels
    \gantttitle{2024}{24} \\
    \gantttitlelist{1,...,12}{2} \\
    % Fases
    \ganttgroup{Fase de Inicio}{1}{4} \\
    \ganttbar{Sprint 1: Plan de Trabajo}{1}{4} \\
    \ganttbar{PEC 1}{4}{4} \\
    \ganttgroup{Fase de Diseño y Análisis}{5}{8} \\
    \ganttbar{Sprint 2 y 3: Análisis y Diseño}{5}{8} \\
    \ganttbar{PEC 2}{8}{8} \\
    \ganttgroup{Fase de Implementación y Pruebas}{9}{12} \\
    \ganttbar{Sprint 4 a 6: Bacula y Pruebas}{9}{12} \\
    \ganttbar{PEC 3}{12}{12} \\
    \ganttgroup{Fase de Documentación}{13}{16} \\
    \ganttbar{Sprint 7 y 8: Memoria y Presentación}{13}{16} \\
    \ganttbar{PEC 4}{16}{16} \\
    \ganttgroup{Fase de Presentación y Defensa}{17}{20} \\
    \ganttbar{Sprint 9: Defensa}{17}{20} \\
    \ganttbar{PV}{20}{20} \\
\end{ganttchart}
\end{landscape}
\bigskip
